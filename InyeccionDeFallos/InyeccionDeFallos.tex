\chapter{Inyección de fallos}
\label{ch:InyeccionDeFallos}

\lettrine[lraise=-0.1, lines=2, loversize=0.2]{L}{a} inyección de fallos es una
técnica que permite recrear los efectos que produce la radiación sobre un
circuito. Esta técnica es ampliamente usada ya que permite estudiar en qué parte
de un circuito causa más efecto un error lógico y qué partes son más resistentes a
este tipo de error. El diseño de circuitos destinados a trabajar en entornos 
hostiles, donde recibirán altas dosis de radiación ionizante, encuentra en esta
técnica una gran ayuda, ya que permite estudiar la sensibilidad de sus módulos a
este tipo de errores sin necesidad de fabricarlo y testearlo bajo radiación real.
Con esto se acelera enormemente el proceso de diseño y refuerzo de circuitos
resistentes a radiación.

% Imagen 1

% Definir run (salidas de un circuito que ha sido inyectado)
% Definir golden run

% Imagen 2
\section{FT-Unshades2}
\label{sec:FT-Unshades2}
% GOLDEN run

\endinput
