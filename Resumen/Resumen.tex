\chapter*{Resumen}
\pagestyle{especial}
\chaptermark{Resumen}
\phantomsection
\addcontentsline{toc}{listasf}{Resumen}

\lettrine[lraise=-0.1, lines=2, loversize=0.2]{E}{l} diagnóstico de \gls{SEU} es
un problema abierto sobre el que apenas se han realizado investigaciones previas.
En este trabajo perseguimos diseñar una nueva técnica de diagnóstico que permita
localizar un \gls{SEU} a partir de la información de la que se disponga.

Es común disponer únicamente de diccionarios de fallos incompletos, ya que, en
circuitos grandes, el tiempo necesario para obtener un diccinario de fallos
completo lo hace inviable. Vamos a ver qué técnica usamos para diagnosticar en
estas situaciones y cuándo se comienza a perder la capacidad de diagnóstico
conforme la exhaustividad del diccionario de reduce.

La hipótesis de la que partimos para diseñar las técnicas de diagnóstico es que
los \gls{SEU} próximos entre si producen patrones similares a la salida. Estos
pueden ser caracterizados de diferentes formas y usados para estimar la
localización real del \gls{SEU} que queremos localizar.

Combinando la información que obtenemos al aplicar distintas métricas sobre la
información disponible, hemos conseguido unos resultados bastante buenos sobre los
diseños en los que se ha probado la técnica. Incluso para aquellos circuitos en
los que no se consigue acertar el biestable y ciclo exactos, la técnica, tras la
primera iteración, acota la localización del \gls{SEU} en un relativamente 
reducido rango de ciclos y a unos registros concretos. A partir de esta primera
acotación podemos obtener un nuevo diccionario de fallos enfocado en las zonas del
circuito señaladas por el algoritmo de diagnóstico y repetir con él el proceso,
mejorando el resultado. El diagnóstico puede darse por finalizado cuando
encontremos un candidatos que produzca exactamente el mismo patrón de salida que
el \gls{SEU} bajo diagnóstico.

Con este proceso iterativo, si el diccionario de partida es lo suficientemente
completo para realizar correctamente la primera estimación, llegará un momento en
el que podamos obtener un diccionario completo de la zona acotada. Si llegados a
este punto aún ha terminado el diagnóstico y las iteraciones han seguido el camino
correcto, el siguiente diccionario contendrá al menos una entrada que cuyo patrón
de salida coincida con el patrón que produce el \gls{SEU} bajo diagnóstico.

Esta técnica puede ser muy útil en el proceso de diseño de circuitos resistentes a
radiación, ya que, por ejemplo, ante cualquier vulnerabilidad encontrada tras
radiar el circuito en el acelerador de partículas, evita repetir el proceso de
diseño completo. Aplicando la técnica se puede saber en que biestable se ha
producido el \gls{SEU} y reforzar la zona en caso de que fuera necesario.


\chapter*{Abstract}
\pagestyle{especial}
\chaptermark{Abstract}
\phantomsection
\addcontentsline{toc}{listasf}{Abstract}

\lettrine[lraise=-0.1, lines=2, loversize=0.2]{}{}

...
\emph{-translation by google-}

