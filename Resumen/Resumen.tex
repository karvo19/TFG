\chapter*{Resumen}
\pagestyle{especial}
\chaptermark{Resumen}
\phantomsection
\addcontentsline{toc}{listasf}{Resumen}

\lettrine[lraise=-0.1, lines=2, loversize=0.2]{E}{l} diagnóstico de
\textit{Conmutaciones por evento único o \gls{SEU}} es un problema abierto sobre 
el que apenas se han realizado investigaciones previas. En este trabajo 
perseguimos diseñar una nueva técnica de diagnóstico que permita localizar un 
\gls{SEU} a partir de la información de la que se disponga.

Es común disponer únicamente de diccionarios de fallos incompletos, ya que, en
circuitos grandes, el tiempo necesario para obtener un diccionario de fallos
completo lo hace inviable. Vamos a ver qué técnica usamos para diagnosticar en
estas situaciones y cuándo se comienza a perder la capacidad de diagnóstico
conforme la exhaustividad del diccionario se reduce.

La hipótesis de la que partimos para diseñar las técnicas de diagnóstico es que
los \gls{SEU} próximos entre si producen patrones de error similares a la salida. 
Estos pueden ser caracterizados de diferentes formas y usados para estimar la
localización real del \gls{SEU} que queremos localizar.

Combinando la información que obtenemos al aplicar distintas métricas sobre la
información disponible, hemos conseguido unos resultados bastante buenos sobre los
diseños en los que se ha probado la técnica. Incluso para aquellos circuitos en
los que no se consigue acertar el biestable y ciclo exactos, la técnica, tras la
primera iteración, acota la localización del \gls{SEU} en un relativamente 
reducido rango de ciclos y a unos registros concretos. A partir de esta primera
acotación podemos obtener un nuevo diccionario de fallos enfocado en las zonas del
circuito señaladas por el algoritmo de diagnóstico y repetir con él el proceso,
mejorando el resultado. El diagnóstico puede darse por finalizado cuando
encontremos un candidato que produzca exactamente el mismo patrón de salida que
el \gls{SEU} bajo diagnóstico.

Con este proceso iterativo, si el diccionario de partida es lo suficientemente
completo para realizar correctamente la primera estimación, llegará un momento en
el que podamos obtener un diccionario completo de la zona acotada. Si llegados a
este punto aún no ha terminado el diagnóstico y las iteraciones han seguido el 
camino correcto, el siguiente diccionario contendrá, al menos, una entrada cuyo 
patrón de salida coincida con el patrón que produce el \gls{SEU} bajo diagnóstico.

Esta técnica puede ser muy útil en el proceso de diseño de circuitos resistentes a
radiación, ya que, por ejemplo, ante cualquier vulnerabilidad encontrada tras
irradiar el circuito en el acelerador de partículas, evita repetir el proceso de
diseño completo. Aplicando la técnica se puede saber en qué biestable se ha
producido el \gls{SEU} y reforzar la zona en caso de que fuera necesario.


\chapter*{Abstract}
\pagestyle{especial}
\chaptermark{Abstract}
\phantomsection
\addcontentsline{toc}{listasf}{Abstract}

\lettrine[lraise=-0.1, lines=2, loversize=0.2]{T}{h}e Single Event Upset (SEU) 
diagnosis is an open problem that has hardly been investigated previously. In this
work we seek to design a new diagnostic technique that allows locating a SEU from 
the information that is available.

It is common to have only incomplete fault dictionaries, since, on large circuits,
the time required to obtain a complete fault dictionary makes it unfeasible. We 
are going to see what technique we use to diagnose in these situations and when 
the diagnostic capacity begins to lose, according to the exhaustiveness of the 
dictionary.

The hypothesis from which we start to design diagnostic techniques is that the 
SEUs close to each other produce similar patterns at the output. These can be 
characterized in different ways and used to estimate the actual location of the 
SEU that we want to locate.

Combining the information we obtain by applying different metrics on the available
information, we have achieved quite good results on the designs in which the 
technique has been tested. Even for those circuits in which it is not possible to 
hit the exact flip-flop and cycle, the technique, after the first iteration, 
limits the location of the SEU in a relatively reduced range of cycles and to 
specific registers. From this first dimension we can obtain a new fault dictionary
focused on the areas of the circuit indicated by the diagnostic algorithm and 
repeat the process with it, improving the result. The diagnosis can be terminated 
when we find a candidate that produces the exact same output pattern as the SEU 
under diagnosis.

With this iterative process, if the starting dictionary is complete enough to make
the first estimate correctly, there will come a time when we can obtain a complete
dictionary of the bounded area. If at this point the diagnosis has not yet finished
and the iterations have followed the correct path, the following dictionary will 
contain at least one entry whose output pattern matches the pattern that the 
diagnostic SEU produces.

This technique can be very useful in the process of designing radiation resistant 
circuits, since, for example, before any vulnerability found after irradiating the 
circuit in the particle accelerator, it avoids repeating the entire design process.
By applying the technique, it is possible to know in which bistable the SEU has 
been produced and to reinforce the area if necessary.



...
\emph{-translation by google-}

