\chapter{Estado del arte}
\label{ch:EstadoDelArte}

% FAULT DETECTION
% Estoy encontrando papers de "fault detection" pero no muchos de "fault location"
\lettrine[lraise=-0.1, lines=2, loversize=0.2]{D}{a}do que no es posible realizar
un diagnóstico de \gls{SEU} sin detectarlo primero, numerosos estudios se centran
en desarrollar técnicas que permitan detectarlos a tiempo para suprimir sus 
efectos. Por ejemplo, en 2014, un equipo chino presentó una técnica de detección 
de \gls{SEU} basada en la \textit{Máquina de Boltzman Restringida o \gls{RBM}}, 
bloque fundamental en muchos algoritmos de \textit{Deep Learning} 
\cite{RBMSEUdetection}. En \cite{SCARA} abordan el problema de \textit{faul 
detection} por el modelo dinámico del sistema. Comparan las lecturas tomadas por
los sensores con los valores teóricos que se obtienen del modelo dinámico del
robot SCARA. De esta forma detectan anomalías debidas a radiación. En un estudio
más reciente, enfocado a sistemas embebidos, emplean programas de detección por
software. Multitud de hilos se ejecutan simultáneamente y se encargan de examinar
el circuito con el objetivo de detectar alguna irregularidad causada por
radiación \cite{DetectingSEUs}.


% FAULT LOCATION
La investigación actual destinada al diagnóstico de \gls{SEU} está menos
desarrollada que la existente para \textit{fault detection}.
% Hablar del paper que usa los bits de configuracion y enlazar con la idea de que
% la mayoria de las veces el CUT es una caja negra donde sólo son accesibles las
% salidas (solo podemos monitorizar las salidas, tenemos que realizar el
% diagnóstico solo a partir de la informacion de las salidas)

    % Leer: An Accurate Fault Location Method Based on Configuration Bitstream
    % Analysis
    \cite{AFLS}


% Bucar en IEEE por "fault location" -> aparecerá diagnostico de fallos de
% fabricacion

% Buscar en IEEE por SEU diagnosis



% Revisar las referencias del paper de Mogollón cita:SEUDiagnosis y hablar de él

% En esta seccion contaría que lo que existe principalmente es para fallos de
% fabricación.

% Que para SEU se ha hecho poquito y basado en el cálculo de códigos hash (papers
% de Mogollón). Explicar que el problema de la técnica basada en códigos hash es
% que necesita de diccionarios exhaustivos, y generarlos es inviable para
% circuitos grandes.

% Dejar claro que el diccionario de una campaña exhaustiva es un diccionario
% exhaustivo o completo, y uno al que le falten "runs" es uno incompleto o no
% exhaustivo.

\endinput
