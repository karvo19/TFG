\chapter{Estado del arte}
\label{ch:EstadoDelArte}

% FAULT DETECTION
% Estoy encontrando papers de "fault detection" pero no muchos de "fault location"
\section{Detección de fallos (\textit{Fault Detection})}
\label{sec:FaultDetection}
Dado que no es posible realizar
un diagnóstico de \gls{SEU} sin detectarlo primero, numerosos estudios se centran
en desarrollar técnicas que permitan detectarlos a tiempo para suprimir sus 
efectos. Por ejemplo, en 2014, un equipo chino presentó una técnica de detección 
de \gls{SEU} basada en la \textit{Máquina de Boltzman Restringida o \gls{RBM}}, 
bloque fundamental en muchos algoritmos de \textit{Deep Learning} 
\cite{RBMSEUdetection}. En \cite{SCARA} abordan el problema de \textit{faul 
detection} por el modelo dinámico del sistema. Comparan las lecturas tomadas por
los sensores con los valores teóricos que se obtienen del modelo dinámico del
robot SCARA. De esta forma detectan anomalías debidas a radiación. En un estudio
más reciente, enfocado a sistemas embebidos, emplean programas de detección por
software. Multitud de hilos se ejecutan simultáneamente y se encargan de examinar
el circuito con el objetivo de detectar alguna irregularidad causada por
radiación \cite{DetectingSEUs}.


% En esta seccion contaría que lo que existe principalmente es para fallos de
% fabricación.
% FAULT LOCATION
\section{Diagnóstico de fallos o localización de fallos}
\label{sec:FaultLocation}
Hasta ahora, el diagnóstico de fallos ha sido poco estudiado, siendo los fallos de
fabricación a los que más esfuerzos de investigación se les ha dedicado
\cite{VLSI, EfficientSA0SA1, RepairSA0SA1, LargeComb, ANewRep, FILC, FDIRC}.
Estos no son el tipo de fallos que nos interesa diagnosticar en esta
investigación, ya que no son causados por radiación, si no que se producen, como
su nombre indica, en el momento de fabricación del circuito (\textit{stuck-at-0,
stuck-at-1}).

Las técnicas existentes para localización de fallos provocados por radiación se
basan principalmente en el uso de diccionarios de fallos, aunque también se
emplean vectores de prueba, listas de fallos, tabla de verdad de nodos
(\textit{"node truth table"}) y tabla de conexiones de pines (\textit{pin
connection table}) \cite{DiagnosisTechniques, LASAR, RTFDandD}. 

A excepción de contados estudios, la mayoría de los revisados modelan al 
circuito bajo prueba o \textit{\gls{CUT}} como una caja negra, es decir, el diseño
del circuito no se conoce y solo las salidas pueden ser monitorizadas.
Normalmente, el número de biestables del circuito es mucho mayor que el número de
salidas, por lo que es necesario observar el circuito el suficiente tiempo como
para detectar patrones que puedan ser asociados a la localización de un
determinado \gls{SEU} \cite{SEUDiagnosis}. Estas huellas son registradas y
almacenadas en un diccionario durante una fase previa al diagnóstico.

El diccionario de fallos se genera mediante inyección de fallos, en alguna
plataforma que lo permita \cite{FastFI, LeonFI, FTU}, y contiene información de 
la localización de los \gls{SEU} inyectados y el patrón de salidas que produce. 
Si el diccionario recoge todas las posibilidades, se habla de diccionario 
completo o exhaustivo, tomando el nombre de la campaña de inyección de fallos 
necesaria para generarlo (\textit{Campaña Exhaustiva}). En el caso contrario, es 
un diccionario incompleto o no exhaustivo, es decir, no todas las posibles 
combinaciones de (biestable, ciclo) han sido inyectadas y almacenadas en el 
diccionario. 

% La localización del \gls{SEU}, una vez detectado, se consigue comparando el 
% patrón de salida generado con la información contenida en los diccionarios.
Durante la fase de diagnóstico, para localizar un \gls{SEU} detectado, se compara
el patrón de error que genera en las salidas del circuito con los patrones
almacenados en el diccionario. Debido al largo tiempo de observación comentado, la
información a comparar puede tener un tamaño considerable, y por tanto el tiempo
necesario para procesar la comparación es alto. Una solución para reducir esta
cantidad de información y por tanto, el tiempo, permitiendo incluso localización
de \gls{SEU} en tiempo real, es comprimirla. Un ejemplo sería el uso de códigos 
HASH \cite{SEUDiagnosis}.

% This premise is pretty ambitious since supposes thatthere  is  an  unique
% signature  for  a  couple  of  clock  cycle  andFF  and  viceversa,  i.e.  this
% relation  is  univocal.  In  general,  itis not true, and several different
% injected faults can give placeto exactly the same signature at the outputs. In
% this scenario,it  is  also  obvious  that  the  greater  the  time  we  analyze
% theoutputs,  the  greater  the  possibility  to  get  different  signaturesfor
% different injected faults.

% Problema de que no se llegue a un único candidato, sino a una lista
% Problema de las colisiones
% There is another problem affecting injectivity related to theCUT  itself.  In  a
% fault  injection  campaign,  when  several  runsare  performed,  it  is
% possible  that  the  CUT  shows  exactly  thesame outputs for different fault
% injections, and not only for theGOLDEN outputs but also for wrong outputs. In
% such cases,the fault dictionary is no longer univocal or unambiguous.
Dada la gran cantidad de biestables existentes en comparación con el reducido
número de salidas, no es dificil imaginar la posibilidad de que dos \gls{SEU}
localizados en biestables y/o ciclos distintos produzcan exactamente el mismo
patrón de error a la salida, al menos durante el tiempo y test programados. 
Cuando esto ocurre, se habla de \textit{"Colisión"}. Además, es posible que un
\gls{SEU} no produzca error alguno a la salida durante el test, siendo
indistinguible de una situación libre de conmutaciones. Ante estas situaciones,
existirá más de una entada del diccionario que conincida con la buscada. Como
resultado del diagnóstico se obtienen no una si no una lista de posibles
localizaciones para el \gls{SEU} bajo diagnóstico.

% Explicar que el problema de la técnica basada en códigos hash es que necesita 
% de diccionarios exhaustivos, y generarlos es inviable para circuitos grandes.
Hasta ahora hemos hablado de diagnóstico empleando diccionarios de fallos
completos, pero si el \gls{CUT} es grande, obtener un diccionario exhaustivo es
una operación inviable, ya que la cantidad de combinaciones biestable-ciclo a
inyectar para ello se vuelve inabarcable. Si se intenta diagnosticar un \gls{SEU}
empleando un diccionario de fallos incompleto, aparecen nuevos problemas, ya que
puede ocurrir que la ubicación correcta no se haya inyectado durante la prueba, y
por tanto no se encuentre en el diccionario. Si además existe una colisión que sí
se ha inyectado, el diagnóstico conlcuirá con una localización única
aparentemente correcta que puede no se acerque nada a la real.


\endinput
