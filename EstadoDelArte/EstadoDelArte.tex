\chapter{Estado del arte}
\label{ch:EstadoDelArte}

% FAULT DETECTION
% Estoy encontrando papers de "fault detection" pero no muchos de "fault location"
\lettrine[lraise=-0.1, lines=2, loversize=0.2]{D}{a}do que no es posible realizar
un diagnóstico de \gls{SEU} sin detectarlo primero, numerosos estudios se centran
en desarrollar técnicas que permitan detectarlos a tiempo para suprimir sus 
efectos. Por ejemplo, en 2014, un equipo chino presentó una técnica de detección 
de \gls{SEU} basada en la \textit{Máquina de Boltzman Restringida o \gls{RBM}}, 
bloque fundamental en muchos algoritmos de \textit{Deep Learning} 
\cite{RBMSEUdetection}. En \cite{SCARA} abordan el problema de \textit{faul 
detection} por el modelo dinámico del sistema. Comparan las lecturas tomadas por
los sensores con los valores teóricos que se obtienen del modelo dinámico del
robot SCARA. De esta forma detectan anomalías debidas a radiación. En un estudio
más reciente, enfocado a sistemas embebidos, emplean programas de detección por
software. Multitud de hilos se ejecutan simultáneamente y se encargan de examinar
el circuito con el objetivo de detectar alguna irregularidad causada por
radiación \cite{DetectingSEUs}.


% En esta seccion contaría que lo que existe principalmente es para fallos de
% fabricación.
% FAULT LOCATION
Hasta ahora, el diagnóstico de fallos ha sido poco estudiado, siendo los fallos de
fabricación a los que más esfuerzos de investigación se les ha dedicado
\cite{VLSI, EfficientSA0SA1, RepairSA0SA1, LargeComb, ANewRep, FILC, FDIRC}.
Estos no son el tipo de fallos que nos interesa diagnosticar en esta
investigación, ya que no son causados por radiación, si no que se producen, como
su nombre indica, en el momento de fabricación del circuito (\textit{stuck-at-0,
stuck-at-1}).

Las técnicas existentes para localización de fallos provocados por radiación se
basan principalmente en el uso de diccionarios de fallos, aunque también se
emplean vectores de prueba, listas de fallos, tabla de verdad de nodos
(\textit{"node truth table"}) y tabla de conexiones de pines (\textit{pin
connection table}) \cite{DiagnosisTechniques, LASAR, RTFDandD}. 

A excepción de algunos pocos, la mayoría de los estudios revisados tratan al 
circuito bajo prueba o \textit{\gls{CUT}} como una caja negra, es decir, el diseño
del circuito no se conoce y solo las salidas pueden ser monitorizadas. Esto limita
mucho la capacidad de diagnóstico, y por eso se necesita recurrir al uso de 
diccionarios de fallos.

Un diccionario de fallos se genera mediante técnicas de inyección de fallos
\cite{}, y
contiene información de la localización de los \gls{SEU} inyectados y el patrón de
salidas que produce. Si el diccionario recoge todas las posibilidades, se habla de
diccionario completo o exhaustivo, tomando el nombre de la campaña de inyección 
de fallos necesaria para generarlo (\textit{Campaña Exhaustiva}). En el caso 
contrario, es un diccionario incompleto o no exhaustivo, es decir, no todas las
posibles combinaciones de (biestable, ciclo) han sido inyectadas y almacenadas en
el diccionario. 

La localización del \gls{SEU}, una vez detectado, se consigue comparando el patrón
de salida generado con la información contenida en los diccionarios de fallos.

% Problema de que no se llegue a un único candidato, sino a una lista
% Problema de las colisiones

% Hablar del paper de mogollon: (creo que es para evitar las colisiones)
\cite{SEUDiagnosis}


% Explicar que el problema de la técnica basada en códigos hash es que necesita 
% de diccionarios exhaustivos, y generarlos es inviable para circuitos grandes.


\endinput
