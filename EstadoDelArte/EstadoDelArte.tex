\chapter{Estado del arte}
\label{ch:EstadoDelArte}

% Estoy encontrando papers de "fault detection" pero no muchos de "fault location"
\lettrine[lraise=-0.1, lines=2, loversize=0.2]{D}{a}do que no es posible realizar
un diagnóstico de \gls{SEU} sin detectarlo primero, numerosos estudios se centran
en esto. Una vez que el \gls{SEU} ha sido detectado, se aplican diferentes
técnicas para suprimir sus efectos. Por ejemplo, en 2014, un equipo chino presentó
una técnica de detección de \gls{SEU} basada en la \textit{Máquina de Boltzman
Restringida o \gls{RBM}}, bloque fundamental en muchos algoritmos de \textit{Deep
Learning} \cite{RBMSEUdetection}. En \cite{SCARA} abordan el problema de
\textit{faul detection} de manera dinámica, comparan las lecturas tomadas de los
sensores con los valores teóricos que se obtienen del modelo dinámico del robot
SCARA. De esta forma detectan anomalías debidas a radiación.


% Bucar en IEEE por "fault location" -> aparecerá diagnostico de fallos de
% fabricacion

% Buscar en IEEE por SEU diagnosis

% Leer: An Accurate Fault Location Method Based on Configuration Bitstream
% Analysis
\cite{AFLS}


% Revisar las referencias del paper de Mogollón cita:SEUDiagnosis y hablar de él

% En esta seccion contaría que lo que existe principalmente es para fallos de
% fabricación.

% Que para SEU se ha hecho poquito y basado en el cálculo de códigos hash (papers
% de Mogollón). Explicar que el problema de la técnica basada en códigos hash es
% que necesita de diccionarios exhaustivos, y generarlos es inviable para
% circuitos grandes.

% Dejar claro que el diccionario de una campaña exhaustiva es un diccionario
% exhaustivo o completo, y uno al que le falten "runs" es uno incompleto o no
% exhaustivo.

\endinput
