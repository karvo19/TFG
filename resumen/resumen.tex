% !TEX root =../LibroTipoETSI.tex
\chapter*{Resumen}
\pagestyle{especial}
\chaptermark{Resumen}
\phantomsection
\addcontentsline{toc}{listasf}{Resumen}

\lettrine[lraise=-0.1, lines=2, loversize=0.2]{E}{n} nuestra Escuela se producen un número considerable de documentos,
tantos docentes como investigadores. Nuestros alumnos también contribuyen a esta producción a través de sus trabajos de
fin de grado, máster y tesis. El objetivo de este material es facilitar la edición de todos estos documentos y a la vez
fomentar nuestra imagen corporativa, facilitando la visibilidad y el reconocimiento de nuestro Centro. \gls{SEU} seu DAC
\gls{ETSI}

%La hoja de estilo utilizada es una versión de la que el Prof. Payán realizó para un libro que desde hace tiempo viene escribiendo para su asignatura. Con ella se han realizado estas notas, a modo de instrucciones, añadiéndole el diseño de la portada. El diseño de la portada está basado en el que el prof. Fernando García García, de nuestra universidad, hiciera para los libros de la sección de publicación de nuestra Escuela.


\chapter*{Abstract}
\pagestyle{especial}
\chaptermark{Abstract}
\phantomsection
\addcontentsline{toc}{listasf}{Abstract}

\lettrine[lraise=-0.1, lines=2, loversize=0.2]{I}{n} our school there are a considerable number of documents, many teachers and researchers. Our students also contribute to this production through its work in order of degree, master's theses. The aim of this material is easier to edit these documents at the same time promote our corporate image, providing visibility and recognition of our Center. 

...
\emph{-translation by google-}

