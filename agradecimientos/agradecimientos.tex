% !TEX root =../LibroTipoETSI.tex
\chapter*{Agradecimientos}
%\pagestyle{especial}
\pagestyle{empty}
%\chaptermark{Agradecimientos}
\phantomsection
%\addcontentsline{toc}{listasf}{Agradecimientos}
%\vspace{1cm}
%{\huge{Agradecimientos}}
%\vspace{1cm}

\lettrine[lraise=-0.1, lines=2, loversize=0.25]{E}{l} diseño de una hoja de estilo en \LaTeX\ para un texto no es en absoluto trivial. Por un lado hay que conocer bien los usos, costumbres y reglas que se emplean a la hora de establecer márgenes, tipos de letras, tamaños de las mismas, títulos, estilos de tablas, y un sinfín de otros aspectos. Por otro, la programación en \LaTeX\ de esta hoja de estilo es muy tediosa, incluida la selección de los mejores paquetes para ello. La hoja de estilo adoptada por nuestra Escuela y utilizada en este texto es una versión de la que el profesor Payán realizó para un libro que desde hace tiempo viene escribiendo para su asignatura. Además, el prof. Payán ha participado de forma decisiva en la adaptación de dicha plantilla a los tres tipos de documentos que se han tenido en cuenta: libro, tesis y proyectos final de carrera, grado o máster. Y también en la redacción de este texto, que sirve de manual para la utilización de estos estilos. Por todo ello, y por hacerlo de forma totalmente desinteresada, la Escuela le está enormemente agradecida.

A esta hoja de estilos se le incluyó unos nuevos diseños de portada. El diseño gráfico de las portadas para proyectos fin de grado, carrera y máster, está basado en el que el prof. Fernando García García, de la Facultad de Bellas Artes de nuestra Universidad, hiciera para los libros, o tesis, de la sección de publicación de nuestra Escuela. Nuestra Escuela le agradece que pusiera su arte y su trabajo, de forma gratuita, a nuestra disposición.

%gradecemos}, a todos nuestros maestros, cuanto nos enseñaron.

{\flushleft{\hfill \emph{Juan José Murillo Fuentes}}}%
\vspace{-.3cm}
{\flushleft{\hfill \emph{Subdirección de Comunicaciones y Recursos Comunes}}}%
{\flushleft{\hfill \emph{Sevilla, 2020}}}%
