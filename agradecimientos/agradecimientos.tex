% !TEX root =../LibroTipoETSI.tex
\chapter*{Agradecimientos}
%\pagestyle{especial}
\pagestyle{empty}
%\chaptermark{Agradecimientos}
\phantomsection
%\addcontentsline{toc}{listasf}{Agradecimientos}
%\vspace{1cm}
%{\huge{Agradecimientos}}
%\vspace{1cm}

\lettrine[lraise=-0.1, lines=2, loversize=0.25]{E}{l} diseño de una hoja de estilo en \LaTeX\ para un texto no es en absoluto trivial. Por un lado hay que conocer bien los usos, costumbres y reglas que se emplean a la hora de establecer márgenes, tipos de letras, tamaños de las mismas, títulos, estilos de tablas, y un sinfín de otros aspectos. Por otro, la programación en \LaTeX\ de esta hoja de estilo es muy tediosa, incluida la selección de los mejores paquetes para ello. La hoja de estilo adoptada por nuestra Escuela y utilizada en este texto es una versión de la que el profesor Payán realizó para un libro que desde hace tiempo viene escribiendo para su asignatura. Además, el prof. Payán ha participado de forma decisiva en la adaptación de dicha plantilla a los tres tipos de documentos que se han tenido en cuenta: libro, tesis y proyectos final de carrera, grado o máster. Y también en la redacción de este texto, que sirve de manual para la utilización de estos estilos. Por todo ello, y por hacerlo de forma totalmente desinteresada, la Escuela le está enormemente agradecida.

A esta hoja de estilos se le incluyó unos nuevos diseños de portada. El diseño gráfico de las portadas para proyectos fin de grado, carrera y máster, está basado en el que el prof. Fernando García García, de la Facultad de Bellas Artes de nuestra Universidad, hiciera para los libros, o tesis, de la sección de publicación de nuestra Escuela. Nuestra Escuela le agradece que pusiera su arte y su trabajo, de forma gratuita, a nuestra disposición.


Orden recomendado:
- Comienza con los agradecimientos más formales, que suelen ir dirigidos a patrocinadores y/o al tutor del proyecto.

- Jerarquiza en función de su influencia en partes relevantes del proyecto, de mayor a menor.

- No uses frases largas, aunque cuando nombres a personas cercanas puedes hacer uso de dedicatorias en el TFG; te dejamos algunos ejemplos de cómo hacerlo más adelante.

- Las dedicatorias en el TFG pueden ser palabras tuyas, propias, o comenzar con un verso, un proverbio, etc.

Algunos ejemplos de dedicatorias:
- … y particularmente agradezco a mi maestro D/Dª ........, por inculcarme el amor por las matemáticas cuando sólo era un niño de 7 años.

- También deseo agradecer el apoyo y la amistad demostrada en todo momento por …...., incluso cuando le llamaba, temeroso de no lograr terminar esta tesis, a altas horas de la madrugada.

- Gracias a mi familia por su amor y apoyo incondicional desde mi nacimiento, que se mantiene siendo un adulto.

- Y deseo agradecer de manera especial al profesor/a ......... de la asignatura ......... porque sin su buen hacer en la docencia no habría sido capaz de acometer el apartado ....... con facilidad.

- La vida es hermosa, y una de las formas en que se manifiesta esta hermosura es en el hecho de poder compartir y disfrutar con quienes amamos, ........., y con quienes nos ayudan en nuestro camino, como han hecho ........ en mi formación académica.



A mis profesores del Colegio Salesiano de Utrera, ...
en especial a mis dos últimos tutores, Dª Elena Ojeda ¿Rodríguez? y D Fernando ¿? ¿? , por la formación que me dieron, pero sobre todo por entenderme, soportarme y apoyarme. Y a D Eduardo Pérez Prados, de quien adquirí mis primeros conocimientos en informática, y quién posteriormente me informó de la existencia de las becas científicas de verano, gracias a las cuales descubrí mi vocación por la robótica, llevándome directamente hasta donde estoy hoy.
%gradecemos}, a todos nuestros maestros, cuanto nos enseñaron.

{\flushleft{\hfill \emph{Álvaro Calvo Matos}}}%
\vspace{-.3cm}
{\flushleft{\hfill \emph{Grado en Ingeniería Electrónica, Robótica y Mecatrónica}}}%
{\flushleft{\hfill \emph{Sevilla, 2020}}}%
