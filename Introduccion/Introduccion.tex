\chapter{Introducción}
\label{ch:Introduccion}

% Introducir la problematica de lo circuitos en el espacio, decir que hay
% radiación ionizante, la cual produce hard error y soft errors, y explicar que es
% un SEU.
\lettrine[lraise=-0.1, lines=2, loversize=0.2]{L}{a} primera vez que se observaron
los efectos de la radiación en satélites en órbita fue a mediados de la década de
1970. Desde entonces, los investigadores han estudiado sus efectos sobre
diferentes circuitos y tecnologías. La radiación puede ser un problema para los
circuitos destinados a trabajar en su presencia. Si esta es ionizante, puede dar
lugar a un \textit{\gls{SEE}}, o un efecto de evento único, provocando un error 
en el circuito. Los daños que provoca la radiación se clasifican en dos grandes 
grupos: \textit{errores físicos ('hard errors')} y \textit{errores lógicos ('soft
errors')}. Las \textit{conmutaciones por evento único o \acrlong{SEU}
(\acrshort{SEU})} son errores lógicos inducidos por radiación en el circuito
que consisten en el cambio de valor de un biestable del mismo. No son daños
permanentes, pero si que puedes afectar al correcto funcionamiento del sistema.

% Importante decir que con la miniaturizacion de los circuitos esto también aplica
% en aviónica y a nivel del mar (referencia "Cosmic radiation comes to ASIC and SOC
% design" de Santarini y otras que encuentre)
Con la miniaturización de los circuitos, la dosis de radiación necesaria para 
provocar un \gls{SEU} es cada vez menor, con la consiguiente aparición de sus 
efectos cada vez a menor altitud \cite{EDN}. Esto acerca el problema de la 
radiación a aplicaciones más comunes como puede ser la aviación o las 
telecomunicaciones. 
Si la radiacion cambia un bit en tu movil no pasa nada, pero si lo hace en un
circuito, fiesta


% Dar razones de por qué puede ser interesante tener cierta capacidad de
% diagnístico de SEU

\endinput
