\chapter{Campañas iterativas a partir de los candidatos seleccionados}
\label{ch:CampanasIterativas}

% Explicar en qué consiste una campaña iterativa. Comentar que puede ser una buena
% solucion para cuando el diccionrio es extremadamente pequeño. -> Random sampling
% y a partir de ahi acotar
\lettrine[lraise=-0.1, lines=2, loversize=0.2]{}{}

% Reducir enormemente el espacio de inyeccion restringiendo a un rango de ciclos,
% registros, todos los regs pero en un único ciclo, un solo FF pero un rango de
% ciclos, etc

% MINIMOS LOCALES
% Explicar la posibilidad de llegar a un minimo local -> soluciones planteadas
    % El uso de los 4 algoritmos simultaneamente -> 4*n candidatos
    % Posibilidad de incluir además un set de inyecciones aleatorias en la
        % siguiente campaña

% LISTA POSIBLES TIPOS DE CAMPAÑA
    % Campaña enfocada en ciclos
    % Campaña enfocada en registros
    % Campaña enfocada en FF concretos
    % Combinación de las anteriores.

% Comentar con qué formato sale la información para la siguiente campaña, siendo
% el usuario el que decide qué tipo de campaña realizar.

\section{Obtención de la lista de candidatos}
\label{sec:Candidatos}
% Mencionar lo ya explicado anteriormente sobre el uso de los 4 algoritmos
% simultaneamente

% Seleccion de n candidatos de cada algoritmo

% Agrupacion de los candidatos en una sola lista (en el siguiente apartado
% explicamos cómo se procesa esta información)

\section{Extracción de la información para la siguiente campaña de inyección de
fallos}
\label{sec:InfoCampana}
% En este punto comprobamos si hay colision, es decir, si el diagnóstico ha
% terminado

% Se agrupan los candidatos reg/FF repetidos y se calcula el rango de ciclos

% Como resultado se obtiene una lista con toda la informacion necesaria para
% realizar la siguiente campaña recomendada

% De las posibilidades antes mencionadas, para las pruebas realizadas se ha
% empleado la opcion de FF y rango de ciclos, siendo a veces el rango de ciclos de
% un solo ciclo realmente.


\section{Resultados experimentales}
\label{sec:IterResults}
% Validacion de los resultados sin necesidad de inyectar (porque conocemos de
% antemano la inyeccion correcta del target run)

% Hipotesis realizadas para validar de esta forma

% Resultados

% Hablar de cuándo se va perdiendo la capacidad de diagnóstico conforme se va
% reduciendo la exhaustividad del diciconario

% Decir que no es lo mismo un diccionario del 1% para el counter que uno del 0'87%
% para la uart porque el núemro de entradas del primer caso es ridículo


\endinput
