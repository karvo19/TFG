\chapter{Campañas iterativas a partir de los candidatos seleccionados}
\label{ch:CampanasIterativas}

% Explicar en qué consiste una campaña iterativa. Comentar que puede ser una buena
% solucion para cuando el diccionrio es extremadamente pequeño. -> Random sampling
% y a partir de ahi acotar
\lettrine[lraise=-0.1, lines=2, loversize=0.2]{L}{as} campañas iterativas serían
el recurso propuesto para diagnósticos que no encuentran candidatos a distancias
cero, es decir, cuando el diccionario no contiene al \gls{SEU} que queremos
diagnósticar. Esta tećnica aplica para diccionarios incompletos, ya que un
diccionario completo siempre contendrá por definición al target run.

Esta técnica se emplea en situaciones en las que el diccionario es extremadamente 
pequeño pero sin llegar a perder completamente la capacidad de diagnóstico. El
concepto es sencillo. Se obtiene un diccionario incompleto del circuito mediante
inyecciones aleatorias (\textit{"Random Samplig"}), a partir de él, realizamos el
primer diagnóstico. Obtenemos una lista preliminar de candidatos a partir de la
cual extraemos la información necesaria para una siguiente inyección, que ya no 
será aleatoria, al menos en su mayoría. El proceso de obtención de diccionario,
diagnóstico, y obtención de un nuevo diccionario a partir del el cual volver a
diagnosticar puede repetirse tantas veces como sea necesario, finalizando con la
detección de al menos una inyección que colisione con el target run.

% Reducir enormemente el espacio de inyeccion restringiendo a un rango de ciclos,
% registros, todos los regs pero en un único ciclo, un solo FF pero un rango de
% ciclos, etc
Según la información de salida del algoritmo que selecciona los candidatos,
podemos realizar nuevas campañas desde distintos enfoques.
% LISTA POSIBLES TIPOS DE CAMPAÑA
    % Campaña enfocada en ciclos
    % Campaña enfocada en registros
    % Campaña enfocada en FF concretos
    % Combinación de las anteriores.
\begin{itemize}
    \item Campaña enfocada en ciclos: si la información de salida no deja duda
        sobre en qué ciclo se localiza el \gls{SEU}, podemos realizar una campaña
        de inyección de fallos en la que sólo inyectemos durante ese ciclo en
        concreto. El espacio total a inyectar se reduce al número de biestables, y
        la longitud de la simulación se divide por la cantidad de ciclos que
        componían cada ejecución.
    \item Campaña enfocada en registros: cuando la información de la salida apunta
        claramente a un registro o un conjunto de ellos, la campaña de inyección
        de fallos a realizar podría centrarse en inyectar esas zonas del circuito,
        reduciendo el espacio de inyección al descartar de la campaña el resto de
        registros del \gls{CUT}.
    \item Campaña enfocada en biestables: del mismo modo, 
    \item Combinación de las restricciones anteriores:
\end{itemize}

% Comentar con qué formato sale la información para la siguiente campaña, siendo
% el usuario el que decide qué tipo de campaña realizar.

% MINIMOS LOCALES
% Explicar la posibilidad de llegar a un minimo local -> soluciones planteadas
    % El uso de los 4 algoritmos simultaneamente -> 4*n candidatos
    % Posibilidad de incluir además un set de inyecciones aleatorias en la
        % siguiente campaña

\section{Obtención de la lista de candidatos}
\label{sec:Candidatos}
% Mencionar lo ya explicado anteriormente sobre el uso de los 4 algoritmos
% simultaneamente

% Seleccion de n candidatos de cada algoritmo

% Agrupacion de los candidatos en una sola lista (en el siguiente apartado
% explicamos cómo se procesa esta información)

\section{Extracción de la información para la siguiente campaña de inyección de
fallos}
\label{sec:InfoCampana}
% En este punto comprobamos si hay colision, es decir, si el diagnóstico ha
% terminado

% Se agrupan los candidatos reg/FF repetidos y se calcula el rango de ciclos

% Como resultado se obtiene una lista con toda la informacion necesaria para
% realizar la siguiente campaña recomendada

% De las posibilidades antes mencionadas, para las pruebas realizadas se ha
% empleado la opcion de FF y rango de ciclos, siendo a veces el rango de ciclos de
% un solo ciclo realmente.


\section{Resultados experimentales}
\label{sec:IterResults}
% Validacion de los resultados sin necesidad de inyectar (porque conocemos de
% antemano la inyeccion correcta del target run)

% Hipotesis realizadas para validar de esta forma

% Resultados

% Hablar de cuándo se va perdiendo la capacidad de diagnóstico conforme se va
% reduciendo la exhaustividad del diciconario

% Decir que no es lo mismo un diccionario del 1% para el counter que uno del 0'87%
% para la uart porque el núemro de entradas del primer caso es ridículo


\endinput
