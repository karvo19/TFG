\chapter{Técnicas de diagnóstico auxiliares}
\label{ch:TecnicasAuxiliares}

\lettrine[lraise=-0.1, lines=2, loversize=0.2]{}{}
% Explicar que se probaron otras distancias y que, como a veces seleccionan buenos
% candidatos que a Lev se le escapan, los algoritmos se han mantenido a modo de
% respaldo.
% Comentar que más adelante, cuando hablemos de las campañas iterativas, su fucion
% además es la de evitar posibles mínimos locales.


\section{Diagnóstico basado en el análisis de imágenes}
\label{sec:HuDist}
% Explicar el concepto. Diagnostico basado en la identificación de patrones con
% técnicas de reconocimientos de objetos en imágenes.
% Conversión de las entradas del diccionario en imágenes
% Explicar que son los momentos invariantes de Hu
% Calculo de distancias en base a los momentos de Hu. Modulo del vector.
% Explicar cómo se seleccionan los candidatos en base a las distancias calculadas


\section{Diagnóstico por coincidencias}
\label{sec:CoincDist}
% Explicar en que consiste
% Explicar cómo se calculan las distancias de coincidencia
% Explicar cómo se seleccionan los candidatos en base a las distancias calculadas


\section{Resultados experimentales}
\label{sec:CycleResults}
% Explicar, en lineas generales, los resultados que se obtienen al realizar una
% selección de candidatos ejecutando estos algoritmos por separado
% Decir que analizarán resultados más adelante.


\endinput
