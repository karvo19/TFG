\chapter{Técnicas de diagnóstico auxiliares}
\label{ch:TecnicasAuxiliares}

\lettrine[lraise=-0.1, lines=2, loversize=0.2]{A}{unque} la distancia de
Levenshtein funciona muy bien una vez llegamos a las soluciones de descartar los 
runs vacíos del diccionario y controlar el número de candidatos seleccionados 
mediante la ordenación y selección de las \textit{"n"} menores distancias en lugar
de establecer un corte según el rango; podemos pensar que si el primer intento ha
dado tan buenos resultados, existirán otras métricas incluso mejores aún por
explorar.

Con este pensamiento en mente, tras convertir las entradas del diccionario en
imágenes binarias, observamos patrones característicos para cada circuito. Cómo
además las inyecciones de los diccionarios exhaustivos estaban ordenadas en la
medida de lo posible, ocurría que a simple vista se podía observar relación entre
una inyección y las que la rodeaban. Podemos observar incluso un efecto tipo
\textit{gif} si pasamos las imágenes ordenadamente con cierta velocidad.

Esta observación corroboraba la hipótesis de partida (hipótesis
\ref{hyp:inicial}), pero además nos llevó a pensar que quizás se podrían aplicar
algoritmos del campo de la percepción al diagnóstico de \gls{SEU}.

En total probamos cuatro distancias diferentes con las que seleccionar candidatos.

\section{Diagnóstico basado en el análisis de imágenes}
\label{sec:HuDist}
% Explicar el concepto. Diagnostico basado en la identificación de patrones con
% técnicas de reconocimientos de objetos en imágenes.
En percepción, el procedimiento estudiado para identificar objetos en
imágenes consistía en binarizar la imagen RGB, aplicar técnicas para que cada
objeto quede separado del resto (los pixeles blancos de cada objeto estén
separados del resto de objetos) y luego caracterizar las imágenes binarias de cada
objeto de forma que puedan ser posteriormente identificados o que incluso se pueda
entrenar un clasificador de objetos en base a ello.

El diagnóstico basado en imágenes consiste en tratar cada run como una imagen, 
y todo lo contenido en una imagen como un único objeto, sin importar que los
fallos produzcan un patrón completamente unido en la imagen o no.

% Conversión de las entradas del diccionario en imágenes
Para realizar la conversión de las entradas del diccionario en imágenes escribimos
un pequeño programa en Python que se encargaba de leer la información del fichero
\textit{"damages.csv"} y generar un \textit{.png} con ella.

% Imágenes de ejemplo
\begin{figure}[htbp]
    \centering
    \subfloat[run 40]{
        \label{subfig:run40}
        \includegraphics
        {TecnicasDeDiagnosticoAuxiliares/figuras/acum40.png}}
    \hfill
    \subfloat[run 42]{
        \label{subfig:run42}
        \includegraphics
        {TecnicasDeDiagnosticoAuxiliares/figuras/acum42.png}}
    \hfill
    \subfloat[run 949]{
        \label{subfig:run949}
        \includegraphics
        {TecnicasDeDiagnosticoAuxiliares/figuras/acum949.png}}
    \hfill
    \subfloat[run 1842]{
        \label{subfig:run1842}
        \includegraphics
        {TecnicasDeDiagnosticoAuxiliares/figuras/acum1842.png}}
    \caption{Visualización de algunas inyecciones del adder\_acum como imágenes}
    \label{fig:AcumRunImagenes}
\end{figure}

\begin{figure}[htbp]
    \centering
    \subfloat[run 32]{
        \label{subfig:run32}
        \includegraphics
        {TecnicasDeDiagnosticoAuxiliares/figuras/counter32.png}}
    \hfill
    \subfloat[run 35]{
        \label{subfig:run35}
        \includegraphics
        {TecnicasDeDiagnosticoAuxiliares/figuras/counter35.png}}
    \hfill
    \subfloat[run 116]{
        \label{subfig:run116}
        \includegraphics
        {TecnicasDeDiagnosticoAuxiliares/figuras/counter116.png}}
    \hfill
    \subfloat[run 401]{
        \label{subfig:run401}
        \includegraphics
        {TecnicasDeDiagnosticoAuxiliares/figuras/counter401.png}}
    \hfill
    \subfloat[run 1185]{
        \label{subfig:run1185}
        \includegraphics
        {TecnicasDeDiagnosticoAuxiliares/figuras/counter1185.png}}
    \hfill
    \subfloat[run 1367]{
        \label{subfig:run1367}
        \includegraphics
        {TecnicasDeDiagnosticoAuxiliares/figuras/counter1367.png}}
    \caption{Visualización de algunas inyecciones del counter como imágenes}
    \label{fig:CounterRunImagenes}
\end{figure}


% Explicar que son los momentos invariantes de Hu
La caracterización escogida para posteriormente identificar a los objetos son los
Momentos invariantes de Hu. Estos presentan la propiedad de ser invariantes a
rotación, traslación y cambios de escala. Esta medida quizás es excesiva dada la 
naturaleza de los patrones que queremos identificar en imágenes, ya que no rotan,
aunque si nos fijamos por ejemplo en la imagen \ref{fig:CounterRunImagenes}
podemos observar cierto cambio de escala.

Los 7 momentos de Hu se calculan a partir de los momentos normales. Sus
expresiones pueden consultarse en \cite{Hu}.
% \begin{equation}
%     \label{eq:Hu1}
%     I_1 = \eta_{20} + \eta_{02}
% \end{equation}
% \begin{equation}
%     \label{eq:Hu2}
%     I_2 = (\eta_{20} - \eta_{02})² + 4 \times \eta²_{11}
% \end{equation}
% \begin{equation}
%     \label{eq:Hu3}
%     I_3 = (\eta_{30} - 3 \times \eta_{12})² + (3 \times \eta_{21} - \eta_{03})²
% \end{equation}
% \begin{equation}
%     \label{eq:Hu4}
%     I_4 = (\eta_{30} + \eta_{12})² + (\eta_{21} + \eta_{03})²
% \end{equation}
% \begin{equation}
%     \label{eq:Hu5}
%     I_5 = (\eta_{30} - 3 \times \eta_{12})(\eta_{30} + \eta_{12})[(\eta_{30} +
%     \eta_{12})² - 3 \times (\eta_{21} + \eta_{03})²] + (3 \times \eta_{21} -
%     \eta_{03})(\eta_{21} + \eta_{03})[3 \times (\eta_{30} + \eta_{12})² -
%     (\eta_{21} + \eta_{03})²]
% \end{equation}

% Calculo de distancias en base a los momentos de Hu. Modulo del vector.
Las distancias basadas en estas expresiones se calculan de forma vectorial. Se
construye un vector de 7 componentes para cada run con sus momentos de Hu; lo
mismo para el target run. La distancia será el módulo de la diferencia entre ambos
vectores.
\begin{equation}
    \label{eq:VectorHu}
    \overrightarrow{Hu_{run}} = (I_1, I_2, I_3, I_4, I_5, I_6, I_7)
\end{equation}
\begin{equation}
    \label{eq:HuDist}
    D_{Hu} = |\overrightarrow{Hu_{run}} - \overrightarrow{Hu_{taget\_run}}|
\end{equation}

% Explicar cómo se seleccionan los candidatos en base a las distancias calculadas
La selección de candidatos en su versión final se realiza como hemos explicado
anteriormente: ordenamos las distancias de Hu de menos a mayor y seleccionamos
como candidatos los \textit{"n"} primeros.

\section{Diagnóstico por coincidencias}
\label{sec:CoincDist}
% Explicar en que consiste
La última de las distancias que probamos fué la que hemos llamado
\textit{"distancia de coincidencia"}. Esta consiste en contabilizar los fallos a
la salida que tienen en común dos inyeciones distintas, ignorando cuántos no. En
la distancia de Levenshtein sucedía todo lo contrario: contabilizábamos las
diferencias ignorando cuánto tenían en común. 

% Explicar cómo se calculan las distancias de coincidencia
Para calcuar las coincidencias sólo hay que realizar una pequeña modificación en
el algoritmo encargado de calcular la distancia de Levenshtein. La única
diferencia es que se emplea la operación lógica \textit{"AND"} en lugar de la
\textit{"XOR"}. La distancia de coincidencia será la inversa del resultado
obtenido tras la suma de todos los bits altos.

% Imagen explicativa del algoritmo:
\begin{figure}[htbp]
    \centering
    \includegraphics[width=0.95\linewidth]
    {TecnicasDeDiagnosticoAuxiliares/figuras/fig63.pdf}
    \caption{Cálculo de la distancia de coincidencia}
    \label{fig:CoincDist}
\end{figure}

% Explicar cómo se seleccionan los candidatos en base a las distancias calculadas
La selección de candidatos se realiza del mismo modo que anteriormente: ordenamos
las distancias de menor a mayor y seleccionamos los \textit{"n"} primeros.


\section{Resultados experimentales}
\label{sec:4Results}
% Explicar, en lineas generales, los resultados que se obtienen al realizar una
% selección de candidatos ejecutando estos algoritmos por separado
Con la finalidad de comprobar si podemos prescindir de algunos de los cuatro
algoritmos como técnica de diagnóstico debido a que alguno de los demás siempre lo
supere, hemos realizado un estudio en el que, para cada circuito, seleccionábamos
100 inyecciones a diagnósticar y calculábamos candidatos independientemente con
las cuatro distancias.

Respecto al método de selección de candidatos explicado hasta ahora, de este
estudio en adelante realizamos modificaciones, de forma que no tienen por qué
seleccionarse \textit{"n"} candidatos exactamete si se dan ciertas situaciones.
\begin{itemize}
    \item Si el diccionario contiene menos de \textit{"n"} runs no vacíos, la
        cantidad de candidatos seleccionados será menor que \textit{"n"} e igual
        a la cantidad de runs no vacíos.
    \item Si, tras ordenar las distancias, al candidato número \textit{"n"} le
        siguen inyecciones equidistantes a él, estas serán también seleccionadas
        cómo candidatos (con un máximo de $2 \times n$ candidatos).
\end{itemize}

El primer caso se explica solo. Respecto al segundo, decidimos ampliar la
selección de esta manera para no descartar candidatos que según la distancia son
tan válidos como el número \textit{"n"}. El tope de $2 \times n$ va en contra de
este razonamiento, pero ocurría que circuitos con muchas colisiones seleccionaban
muchísimos candidatos.

% Tablas
\begin{table}[htbp]
    \ttabbox
    {\caption{Resultados experimentales. Diagnóstico del registro de inyección con
    los 4 algoritmos. Dicccionarios incompletos ($\leq5\%$)}
    \label{tab:ResReg}}
    {
        \begin{tabular}{c|c c c c|c}
            \hline
            \rule[-8pt]{0pt}{22pt}{\bfseries{Diseños\textbackslash{}Algoritmos}}
            &{\bfseries{Levenshtein}}
            &{\bfseries{Ciclos}}
            &{\bfseries{Hu}}
            &{\bfseries{Coincidencias}}
            &{\bfseries{Todos}}\\
            \hline
            \rule{0pt}{14pt}
            adder\_acum & 100 & 100 & 100 & 100 & 100\\
            counter & 100 & 100 & 100 & 100 & 100\\
            dual\_counter & 99 & 98 & 99 & 100 & 100\\
            fifo & 98 & 98 & 97 & 95 & 99\\
            fir\_ri (37'78\%) & 29 & 21 & 55 & 30 & 78\\
            pcm & 82 & 67 & 82 & 73 & 82\\
            shiftreg & 100 & 100 & 100 & 32 & 100\\
            simple\_fsm & 100 & 100 & 100 & 100 & 100\\
            uart (0'87\%) & 97 & 92 & 96 & 84 & 97\\
            \hline
        \end{tabular}
    }
\end{table}
\begin{table}[htbp]
    \ttabbox
    {\caption{Resultados experimentales. Diagnóstico del FF de inyección con
    los 4 algoritmos. Dicccionarios incompletos ($\leq5\%$)}
    \label{tab:ResFF}}
    {
        \begin{tabular}{c|c c c c|c}
            \hline
            \rule[-8pt]{0pt}{22pt}{\bfseries{Diseños\textbackslash{}Algoritmos}}
            &{\bfseries{Levenshtein}}
            &{\bfseries{Ciclos}}
            &{\bfseries{Hu}}
            &{\bfseries{Coincidencias}}
            &{\bfseries{Todos}}\\
            \hline
            \rule{0pt}{14pt}
            adder\_acum & 98 & 37 & 35 & 100 & 100\\
            counter & 96 & 63 & 72 & 100 & 100\\
            dual\_counter & 99 & 67 & 74 & 94 & 100\\
            fifo & 1 & 0 & 0 & 1 & 1\\
            fir\_ri (37'78\%) & 3 & 4 & 4 & 10 & 16\\
            pcm & 69 & 52 & 68 & 56 & 69\\
            shiftreg & 82 & 52 & 53 & 4 & 91\\
            simple\_fsm & 88 & 88 & 88 & 88 & 88\\
            uart (0'87\%) & 93 & 74 & 91 & 79 & 93\\
            \hline
        \end{tabular}
    }
\end{table}

% Estamos seleccionando un número muy bajo de candidatos, 5, es decir, 20
% candidatos en condiciones normales, muchos de los cuales serán el mismo
% seleccionado por diferentes alforitmos.
El primer detalle a destacar de ambas tablas (\ref{tab:ResReg} y \ref{tab:ResFF}) 
es el elevado números de veces que se aciertan tanto el registro como el \gls{FF} 
de inyección agrupando los candidatos de los cuatro algoritmos y seleccionando 
únicamente 5 candidatos con cada uno. Cabe resaltar que en las situaciones 
concretas mencionadas, pueden ser más o menos de 5 candidatos por algoritmo. 
También es muy posible que los candidatos estén repetidos, ya que un buen 
candidato tendrá simultáneamente distancias pequeñas en todos los algoritmos, 
con el límite de 0 para las colisiones (excepto la distancia de coincidencia, 
que funciona de forma diferente).

% Comparar si unos son  mejores que otros
Como habíamos comentado anteriormente, los datos parecen indicar que la distancia
de Levenshtein siempre tiene más acierto que la distancia en ciclos. Sin embargo,
esto no siempre se cumple con el resto. El primer ejemplo de ello es el
\textit{dual\_counter} de la tabla \ref{tab:ResReg}, donde por primera vez el
algoritmo de Levenshtein es superado por el de coincidencias. Esto mismo vuelve a
suceder de forma más acusada en la tabla \ref{tab:ResFF} con los algoritmos de Hu
y Coincidencia.

% Son especialmente interesante las filas de la tabla cuya casilla "Todos" es
% mayor que las demás -> explicar por qué
Son especialmente interesante las filas donde la columna \textit{"Todos"} es
mayor estricto al resto. El número contenido en la columna \textit{"Todos"}
representa el número de veces que obtenemos el registro y \gls{FF} correcto con
los cuatro algoritmos conjuntamente. El hecho de que esta columna a veces sea 
mayor que las anteriores demuestra que no hay algoritmo mejor, sino que
dependiendo del caso concreto, funcionará mejor un algoritmo u otro. Los casos más
claros son el \textit{fir\_ri} de la tabla \ref{tab:ResReg} y los casos
\textit{fir\_ri} y \textit{shiftreg}  de la tabla \ref{tab:ResFF}.

Cabe mencionar que los resultados obtenidos con el \textit{simple\_fsm} no son
válidos, ya que el diccionario exhaustivo tiene 64 entradas y este fue reducido
hasta un diccionario del 5\% de exhaustividad, haciendo un total de 5 entradas de
las cuales 1 además es vacía. Dado que la \textit{"n"} se fijó a 5, los algoritmos
estaban seleccionando siempre los únicos 4 candidatos disponibles. Esto lo supimos
posteriormente a obtener los resultados de los experimentos.

Así mismo, comprobamos que los malos resultados obtenidos tanto en \textit{fifo}, 
como en \textit{fir\_ri}, como en \textit{pcm}, únicos circuitos en los que el
diagnóstico se aleja del 90\% de aciertos espaciales, se deben a la presencia de
un alto número de colisiones. Dada la limitación de $2 \times n$ candidatos
impuesta para runs equidistantes, los candidatos correctos están permaneciendo
fuera de la selección.

% Explicar que se probaron otras distancias y que, como a veces seleccionan buenos
% candidatos que a Lev se le escapan, los algoritmos se han mantenido a modo de
% respaldo.
En el siguiente capítulo explicaremos en que consiste una \textit{Campaña
Iterativa}. Para realizar la primera iteración, es necesario realizar también un
primer diagnóstico. Basándonos en  los resultados presentes en las tablas 
\ref{tab:ResReg} y \ref{tab:ResFF}, hemos decidido seleccionar $4 \times n$
candidatos, es decir, \textit{"n"} candidatos con cada algoritmo (cantidad
variable en función de las situaciones especiales comentadas). Es de esperar que
ciertos candidatos aparezcan repetidos, cada vez que esto ocurra se contabilizará,
siendo una medida extra de la calidad de cada uno de los candidatos, ya que habrá
sido seleccionado por más de un algoritmo.

% Comentar que más adelante, cuando hablemos de las campañas iterativas, su fucion
% además es la de evitar posibles mínimos locales.
Como valor añadido, la selección simultánea de candidatos con varias distancias
puede ser una posible medida para evitar mínimos locales. El algoritmo que mejor
encaja como forma de abordar los mínimos locales es el de la distancia temporal,
ya que, como concluimos anteriormente, selecciona inyecciones que desde el punto
de vista de los biestables, con excepciones, son más aleatorias. Abordaremos con 
más detalle el problema de los mínimos locales en el capítulo 
\ref{ch:CampanasIterativas}.

% La selección de candidatos por distancia de ciclo, aunque en principio no tengan
% capacidad de diagnóstico por si solo (con excepciones), es una forma de
% seleccionar circuitos de localizacion espacial aleatorio cuando no nos
% encontramos en una de las excepciones. (Justificación extra para la distancia
% temporal)

% Decir que los resultados de los 4 algoritmos se convinarán más adelante en las
% campañas interativas.

\endinput
