\chapter{Inclusión de la distancia temporal en el algoritmo de selección de
candidatos}
\label{ch:Cycle}

% Explicar el concepto de la distancia temporal con algún ejemplo.
\lettrine[lraise=-0.1, lines=2, loversize=0.2]{L}{a} idea de implementar una
distancia basada en ciclos surge de la observación realizada en los resultados de
la distancia de Levenshtein (ver subsección \ref{subsec:LevDicNoExhaust}). La
también llamada \textit{"distancia de ciclos"} se calcula a partir del primer 
ciclo en que la inyección se manifiesta a la salida, a partir de ahora 
\textit{"first cycle"}. La distancia de ciclos se calcula como la diferencia 
entre el \textit{first cycle} del target run y el \textit{first cycle} de cada 
entrada del diccionario, pudiendo ser positiva, negativa o cero si se los fallos 
si el fallo del target run se manifiesta antes, despues o en el mismo ciclo que 
las entradas del diccionario.

% Decir las hipótesis realizadas para diagnosticar en base a la distancia de Cycle
    % Esta distancia no es nada sin la de levenshtein. Lev = 0 es que hemos i
    % terminado, cycle = 0 no implica Lev = 0
La hipótesis que realizamos basándonos en las observaciones y la cual aplicamos
para mejorar el diangóstico sería la siguiente:
\begin{hypothesis}\label{hyp:cycle}
    "La distancia temporal, definida como la diferencia entre los primeros ciclos
    en que se manifiestan dos inyecciones, guarda relación directa con la
    diferencia entre los ciclos de inyección de dichas inyecciones".
\end{hypothesis}



% Explicar que la base de datos de distancias se genera de forma similar

% Decisiones tomada cuando el run es vacío y por qué


\section{Diagnóstico basado en la distancia temporal}
\label{sec:CycleCands}
% Explicar las tres situaciones posibles que pueden darse:
    % Distancia positiva
    % Distancia negativa
    % Distancia nula: dos posibilidades
        % El SEU no produce fallo a la salida
        % Aun no se ha manifestado el fallo
% Explicar cómo se seleccionan los candidatos en base a las distancias calculadas


\section{Fusión de las distancias temporal y de Levenshtein}
\label{sec:FusionLevenCycle}
% Explicar como se modifica el algotirmo para fusionar las dos distancias


\section{Resultados experimentales}
\label{sec:CycleResults}


\subsection{Diccionarios exhaustivos}
\label{subsec:CycDicExhaust}


\subsection{Diccionarios no exhaustivos}
\label{subsec:CycDicNoExhaust}


\endinput
