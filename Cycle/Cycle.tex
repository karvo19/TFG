\chapter{Inclusión de la distancia temporal en el algoritmo de selección de
candidatos}
\label{ch:Cycle}

\lettrine[lraise=-0.1, lines=2, loversize=0.2]{}{}
% Explicar el concepto de la distancia temporal con algún ejemplo.
% Decir las hipótesis realizadas para diagnosticar en base a la distancia de Cycle
% Explicar que la base de datos de distancias se genera de forma similar


\section{Diagnóstico basado en la distancia temporal}
\label{sec:CycleCands}
% Explicar las tres situaciones posibles que pueden darse:
    % Distancia positiva
    % Distancia negativa
    % Distancia nula: dos posibilidades
        % El SEU no produce fallo a la salida
        % Aun no se ha manifestado el fallo
% Explicar cómo se seleccionan los candidatos en base a las distancias calculadas


\section{Fusión de las distancias temporal y de Levenshtein}
\label{sec:FusionLevenCycle}
% Explicar como se modifica el algotirmo para fusionar las dos distancias


\section{Resultados experimentales}
\label{sec:CycleResults}


\subsection{Diccionarios exhaustivos}
\label{subsec:CycDicExhaust}


\subsection{Diccionarios no exhaustivos}
\label{subsec:CycDicNoExhaust}


\endinput
