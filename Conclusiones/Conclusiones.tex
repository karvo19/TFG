\chapter{Conclusiones y trabajos futuros}
\label{ch:FFdist}
% Las conclusiones en formato:
    % Se ha hecho X, Y, y funciona muy bien.
    % Se ha visto que ocurre A, B, C

\section{Conclusiones}
\label{sec:Conclusiones}
El propósito de esta investigación era desarrollar técnicas de diagnóstico de
\gls{SEU} utilizando diccionario de fallos incompletos. Para ello, partíamos de la
hipótesis \ref{hyp:inicial}, marcando la línea de investigación en encontrar una
métrica adecuada que permita sacar partido de las similitudes en los patrones que
generan dos conmutaciones lógicas próximas entre sí.

Hemos realizado experimentos en los que se diagnosticaba aplicando métricas cuyos
enfoques eran diferentes, todos ellos devolviendo unos resultados muy buenos.
Además, hemos establecido las bases de una técnica de diagnóstico consistente en
la realización de numerosas iteraciones de \textit{"Campaña de inyección de 
fallos para generar un diccionario de fallos"}, \textit{"Diagnóstico empleando el
diccionario de fallos obtenido"}, \textit{"Extracción de candidatos que acoten el
espacio de inyección"}, con la cuál hemos podido realizar diagnósticos exitosos 
con diccionarios de fallos muy incompletos en circuitos de grandes dimensiones.
Además, observamos como en muchas ocasiones, el diagnóstico se resuelve con la
campaña inicial, ya que los candidatos obtenidos no dejan duda.

Se ha propuesto además que siempre se realice una campaña extra tras finalizar el
diagnóstico, ya que los resultados apuntan a que iteraciones extras obtendrían, de
darse el caso, otras colisiones extras que enriquecerían el diagnóstico.

La técnica de campañas iterativas puede llevarnos hacia un mínimo local donde no
se encuentre el \gls{SEU} bajo diagnóstico, fallando su misión. Hemos propuesto
soluciones a este problema tales como la inclusión de inyecciones aleatorias en
cada iteración.

Hemos comprobado que la distancia temporal es especialmente útil para localizar
temporalmente el \gls{SEU} bajo diagnóstico. Para circuitos grandes, con
diccionarios muy reducidos, la exactitud de la distancia en ciclos se reduce, pero
los resultados demuestran que es capaz de acotar bastante la localización temporal
del \gls{SEU}.

La distancia de Levenshtein funciona especialmente bien como métrica de
diagnóstico. Hemos comprobado que a distancias de Levenshtein bajas, el cálculo de
ciclo realizado con la distancia temporal (ecuación \ref{eq:CicloCentral}) mejora,
aunque no es un requisito imprescindible para que este cálculo funcione con
exactitud.

% Conclusion sobre la perdida de diagnóstico con la exhaustividad:
    % Mas que exhaustividad, importa el número de entradas del diccionario
    % Se observa como la capacidad de diagnóstico se pierde progresivamente,
        % aunque por falta de experimentos no se puede asegurar en todos los
        % casos.
Los experimentos encaminados a determinar cuándo se pierde la capacidad de
diagnóstico sobre un circuito a base de reducir progresivamente la exhaustividad
del diccionario muestran que ésta no se pierde instantáneamente. Primero perdemos
la capacidad de localizar el biestable exacto, y posteriormente se van encontrando
más dificultades para determinar el registro donde se localiza el \gls{SEU},
siendo la capacidad de diagnóstico temporal lo último que se pierde. Es en este
punto de pérdida total donde deja de tener potencial de diagnóstico completo la
técnica iterativa, ya que hasta entonces, existirán campañas que mejoren el
resultado.

El hecho de que el primer intento de encontrar una distancia útil para diagnóstico
haya resultado muy bien, nos lleva a pensar que podrían existir muchas otras
métricas aún por explorar y que funcionen incluso mejor.
Podemos concluir la investigación afirmando que estas técnicas tienen capacidad de
diagnóstico con diccionarios de fallos incompletos; especialmente la técnica de 
campañas iterativas, la cual mantiene capacidad de diagnóstico incluso con 
diccionarios de fallos muy incompletos.

\section{Trabajos futuros}
\label{sec:TrabajosFuturos}
% Mejora: tabla dispersa para optimizar memoria
% Realizar el programa en C para acelerar el proceso
Las mejoras más inmediatas que se podrían realizar sobre la técnica son
optimizaciones computacionales. El código esta realizado completamente en Python,
por ser un lenguaje de programación con el que se hacen scripts muy rápidamente.
Traducir el lenguaje a código C reduciría significativamente el coste
computacional y los tiempos de cómputo.

Para optimizar el uso de la memoria, se podría reemplazar la forma de almacenar
los daños de cada entrada del diccionario por tablas dispersas.

% Aplicar la técnica en circuitos de mayor complejidad: edelweiss y 8051.
% Realizar experimentos reales, probar la técnica bajo radiación
Como trabajo futuro más inmediato, se podría aplicar la técnica a circuitos de
mayor extensión. Además sería interesante aplicar la técnica en circuitos tras
realizarles tests de radiación.

Por último, se podrían investigar otras formas de explotar las similitudes
existentes entre patrones cercanos.

\subsection{Otras técnicas de tratamiento de imágenes}
\label{subsec:OtrasTecnicasImag}
% Decir que existen otras muchas opciones para la localización de SEU empleando
% técnicas de diagnóstico de imágenes aún por explorar
El campo de la percepción por ordenador está muy desarrollado actualmente y avanza
muy rápido. Sería interesante explorar qué otras opciones existen para
diagnosticar basándonos en el tratamiento de imágenes.

% Estudiar otras formas mejores de indentificar patrones en imágenes para este
% caso

% Mencionar algunos ejemplos

% Mencionar avances del deep learning en el campo de la percepción que podrían
% impulsar el diagnóstico por esta vía.
Además, el \textit{deep learning} está experimentando grandes avances que podrían ser
aplicados para entrenar una red neuronal capaz de localizar un \gls{SEU} del
circuito para la que se le ha entrenado.


\subsection{Distancia en flip-flops. Mejora de la distancia temporal}
\label{subsec:FFdist}
Una posible mejora del diagnóstico, basada en la topología del circuito, puede ser
la que hemos llamado \textit{"Distancia en FF"}.

% Explicar el concepto de distancia en FF
La distancia en \gls{FF} de un biestable a otro sería el número mínimo de ciclos
de reloj que tomaría propagar un \gls{SEU} de uno a otro. Esta distancia puede
verse afectada por la dirección: 

\begin{center}
    $D(A-B) \neq D(B-A)$
\end{center}

Calculamos que esta distancia puede ser útil para afinar el diagnóstico una vez ya
se haya acotado lo suficiente al \gls{SEU}.

% Explicar como se calcula la distancia en FF
Para calcular la distancia en FF habría que contar cuántos elementos de lógica
combinacional separan dos biestables. Esto podría realizarse empleando la suite de 
Yosys por ejemplo \cite{YOSYS}.

\endinput
