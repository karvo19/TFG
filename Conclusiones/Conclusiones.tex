\chapter{Conclusiones y trabajos futuros}
\label{ch:FFdist}


\section{Conclusiones}
\label{sec:Conclusiones}
% Anotaciones en el indice. Ver indice (el que hice con poli)
% Ver también OneNote/DistanciasCombinadas/Observaciones

% A falta de probar el algoritmo con ciruitos de mayor complejidad y tamaño, como
% pueden ser edelweiss y 8051, los resulados han sido bastante satifactorios

% Conclusion sobre la perdida de diagnóstico con la exhaustividad:
    % Mas que exhaustividad, importa el número de entradas del diccionario
    % Se observa como la capacidad de diagnóstico se pierde progresivamente,
        % aunque por falta de experimentos no se puede asegurar en todos los
        % casos.

\section{Trabajos futuros}
\label{sec:TrabajosFuturos}
% Anotaciones en el índice. Ver indice (el que hice con poli)
% Ver tambien OneNote/PosiblesFuturasMejoras/Listadeposiblesmejoras

% Mejora: tabla dispersa para optimizar memoria
% Realizar el programa en C para acelerar el proceso

\subsection{Otras técnicas de tratamiento de imágenes}
\label{subsec:OtrasTecnicasImag}
% Decir que existen otras muchas opciones para la localización de SEU empleando
% técnicas de diagnóstico de imágenes aún por explorar

% Estudiar otras formas mejores de indentificar patrones en imágenes para este
% caso

% Mencionar algunos ejemplos

% Mencionar avances del deep learning en el campo de la percepción que podrían
% impulsar el diagnóstico por esta vía.

\subsection{Distancia en flip-flops. Mejora de la distancia temporal}
\label{subsec:FFdist}
% Explicar el concepto de distancia en FF

% Explicar como se calcula la distancia en FF

\endinput
