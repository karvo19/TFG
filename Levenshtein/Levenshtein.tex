\chapter{Primera aproximación a una métrica apropiada. Distancia de Levenshtein}
\label{ch:Levenshtein}

\lettrine[lraise=-0.1, lines=2, loversize=0.2]{L}{a} distancia de Levenshtein
recibe su nombre del científico ruso Vladimir Levenshtein, quién la creó en 1965.

% Definir la distancia de Levenshtein original (citarlo bien)
\vspace{0.3cm}
\textit{"La distancia de Levenshtein, distancia de edición o distancia entre
palabras es el número mínimo de operaciones requeridas para transformar una cadena
de caracteres en otra"}.
\vspace{-0.2cm}
{\flushleft{\hfill \emph{- Wikipedia, 2020, párrafo 1,} \cite{Wikipedia}}}
\vspace{0.3cm}

% Explicar la modificación realizada para adaptarlo a nuestro caso
Aunque este algoritmo esté concebido como métrica de la diferencia entre dos
cadenas de caracteres, podemos aplicar el concepto básico a dos salidas de la
campaña de inyección de fallos. Redefiniendo la distancia de Levenshtein para
nuestro caso en particular:

\vspace{0.3cm}
\textit{"La distancia de Levenshtein es el número mínimo de bits que hay que
conmutar de la salida de una campaña de inyección de fallos para transformarla 
en otra"}.
\vspace{0.3cm}

Existe una operación lógica ya mencionada anteriormente que permite comparar dos
salidas obteniendo como resultado ceros para aquellos bits en los que son iguales
y unos en los diferentes. La operación \textit{XOR} bit a bit permite obtener
tantos unos como diferencias existen entre las dos salidas. La distancia de
levenshtein entre dos salidas de una campaña es el sumatorio de todos estos bits 
con valor lógico alto.

% Comentar la hipótesis realizada para diagnosticar en base a esto
    % Dos SEU proximos entre sí provocarán patrones de error similares a la salida
De esta forma, según la hipótesis inicial \ref{hyp:inicial}, dos \gls{SEU}
que estén próximos entre sí tendrán una distancia de Levenshtein relativamente
baja entre ellos.
    % Si el SEU se encuentra en el diccionario -> levenDist = 0
    % Si el SEU no se encuentra en el diccionario -> min(levenDist) será el más
    % cercano

%\section{Distancia de Levenshtein}
%\label{sec:Levenshtein}

\section{Elaboración de la base de datos de distancias}
\label{sec:LevenDist}
% Explicar como se calculan las distancias de Leven y como se almacenan en memoria
    % Hex -> binario almacenado como enteros -> XOR bit a bit -> sumatorio de 1s
    % Esto para cada entrada del diccionario -> se obtiene tabla simetrica de
        % distancias respecto a la diagonal.

\section{Diagnóstico basado en la distancia de Levenshtein}
\label{sec:LevenCands}
% Para el diagnóstico no se calcula la tabla de distancias completa, sino que se
% calculan solo las del target run con el resto de entradas del diccionario para
% ahorrar memoria.

% Explicar cómo se seleccionan los candidatos en base a las distancias calculadas
% y por qué se selecciona de esa manera y no de otra. En one note hay apuntes de
% esto


\section{Resultados experimentales}
\label{sec:LevenResults}


\subsection{Diccionarios exhaustivos}
\label{subsec:LevDicExhaust}
% Siempre aparece mínimo el correcto, a veces más debido a colisiones


\subsection{Diccionarios no exhaustivos}
\label{subsec:LevDicNoExhaust}


\endinput
