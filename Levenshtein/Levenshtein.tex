\chapter{Primera aproximación a una métrica apropiada. Distancia de Levenshtein}
\label{ch:Levenshtein}

\lettrine[lraise=-0.1, lines=2, loversize=0.2]{L}{a} distancia de Levenshtein
recibe su nombre del científico ruso Vladimir Levenshtein, quién la creó en 1965.

% Definir la distancia de Levenshtein original (citarlo bien)
\vspace{0.3cm}
\textit{"La distancia de Levenshtein, distancia de edición o distancia entre
palabras es el número mínimo de operaciones requeridas para transformar una cadena
de caracteres en otra"}.
\vspace{-0.2cm}
{\flushleft{\hfill \emph{- Wikipedia, 2020, párrafo 1,} \cite{Wikipedia}}}
\vspace{0.3cm}

% Explicar la modificación realizada para adaptarlo a nuestro caso
Aunque este algoritmo esté concebido como métrica de la diferencia entre dos
cadenas de caracteres, podemos aplicar el concepto básico a dos salidas de la
campaña de inyección de fallos. Redefiniendo la distancia de Levenshtein para
nuestro caso en particular:

\vspace{0.3cm}
\textit{"La distancia de Levenshtein en el número mínimo de bits que hay que
conmutar de la salida de una campaña de inyección de fallos para transformarla 
en otra"}.
\vspace{0.3cm}
% Comentar la hipótesis realizada para realizar el diagnóstico en base a esto
    % Si el SEU se encuentra en el diccionario -> levenDist = 0
    % Si el SEU no se encuentra en el diccionario -> min(levenDist) será el más
    % cercano

%\section{Distancia de Levenshtein}
%\label{sec:Levenshtein}

\section{Elaboración de la base de datos de distancias}
\label{sec:LevenDist}
% Explicar como se calculan las distancias de Leven y como se almacenan en mem


\section{Diagnóstico basado en la distancia de Levenshtein}
\label{sec:LevenCands}
% Explicar cómo se seleccionan los candidatos en base a las distancias calculadas
% y por qué se selecciona de esa manera y no de otra. En one note hay apuntes de
% esto


\section{Resultados experimentales}
\label{sec:LevenResults}


\subsection{Diccionarios exhaustivos}
\label{subsec:LevDicExhaust}


\subsection{Diccionarios no exhaustivos}
\label{subsec:LevDicNoExhaust}


\endinput
