\chapter{Primera aproximación a una métrica apropiada. Distancia de Levenshtein}
\label{ch:Levenshtein}

\lettrine[lraise=-0.1, lines=2, loversize=0.2]{}{}
% Definir la distancia de Levenshtein original (citarlo bien)
% Explicar la modificación realizada para adaptarlo a nuestro caso
% Comentar la hipótesis realizada para realizar el diagnóstico en base a esto


%\section{Distancia de Levenshtein}
%\label{sec:Levenshtein}

\section{Elaboración de la base de datos de distancias}
\label{sec:LevenDist}
% Explicar como se calculan las distancias de Leven y como se almacenan en mem


\section{Diagnóstico basado en la distancia de Levenshtein}
\label{sec:LevenCands}
% Explicar cómo se seleccionan los candidatos en base a las distancias calculadas
% y por qué se selecciona de esa manera y no de otra. En one note hay apuntes de
% esto


\section{Resultados experimentales}
\label{sec:LevenResults}


\subsection{Diccionarios exhaustivos}
\label{subsec:LevDicExhaust}


\subsection{Diccionarios no exhaustivos}
\label{subsec:LevDicNoExhaust}


\endinput
